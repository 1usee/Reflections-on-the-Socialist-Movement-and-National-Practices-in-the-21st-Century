\documentclass{article}
\usepackage{ctex}
\begin{document}
\section{保护我们的现代生活}
当我们的生活已经被无数工业品充满的时候,我们并没有感受到幸福的降临。相反,焦虑和日益膨胀的占有欲代替了被许诺的平静幸福的生活。
消费主义的崛起不是一朝一夕的事情,分配所得不能够满足过度消费的欲望也是被充分讨论的,与消费主义相对立的长期主义也有一定的基础,为何现在依旧是消费主义的天下呢?
必须指出,消费是一个简单获取快感的途径,在付出了金钱之后,商品被有保障的送到了消费者手中,无可争辩,无法质疑。当“消费”这一行为脱离了原本作为生存必须的行为,转而变成彰显自身力量的娱乐行为的时候,我们必须重新审视我们的现代生活所追求的究竟是什么?
资本主义的社会生活中,广大劳动者通过自己的劳动换取用于谋生的物质,而资产阶级的剥削所得远远超过了他们实际需要的。就如同封建时代的贵族一样,资产阶级开始展示与无产阶级的差异化,一开始是奢侈品。但是当无产阶级的知识不断积累,广大的劳动者发现了追求奢侈品与资本主义追求的增长之间的矛盾,资产阶级的叙事出现了裂痕。于是新的包装出现了:资产阶级开始将自己的生活包装成精致的生产活动,似乎资产阶级只是在精致的环境中劳动,并为自己赚取了海量的利润。终于,资产阶级的表现与资产阶级的精神变得一致了,资产阶级让所有人以为利润是由他们的双手挣得的了。
这样的假象让无数劳动者开始压榨自己——搭建一个又一个永远不会使用的“工作流”,学习一个又一个根本不感兴趣的知识。甚至连人类的本能——爱本能都被异化了,人们竟然在视频平台上学习如何恋爱!我们对于恋爱的概念被彻底的扭曲为性本能与经济利益的媾和。

\end{document}