\documentclass{article}
\usepackage{ctex}
\begin{document}
\section{start}
二十一世纪的社会主义运动
在二十一世纪20年代,资本主义的内在联系逐渐脱离早期的管理与生产,转向名义所有权的争夺。资本主义社会的成熟意味着资本家的利益被资本主义法律很好的保护着。所以股票、期货,这些所有权的代表物就在社会中流通起来了。但是,当所有权发生变动的时候,在一线的劳工,那些真正付出劳动的人,他们的处境并不会发生本质的改变,这是资本主义经济的所有权与经营权的分离导致的结果。
随着资本主义社会的逐渐成熟,尤其是在如何对付无产阶级运动方面的成熟,我们清晰的意识到:想要以尽可能小的代价去终结资本主义就必须从经济方面下手,要避开与资本主义社会的正面武装冲突,从生产方面瓦解资本主义经济。
在二十一世纪,科技的发展使得人们学习的成本大幅的下降,尤其是AI的出现,人们可以快速的整合过去大量、重复的资料。3D打印使得人们可以按照自己的想法从设计零件开始按照自己的想法构建自己的工业产品。在这个资本主义充分发展的时代,想要彻底推翻资本主义,就只能从生产侧下手——通过开源平台让大众掌握生产能力,从经济上证明资本主义的不必要。这样就可以通过尽可能小的代价动摇资本主义。
1.	社会主义运动的条件
纵观历史,任何革命发生的前提都是现行经济体制的崩坏。因此,社会主义革命的机会就在经济危机时期。经济危机时期容易获得群众支持,容易筹备革命力量,宣传阻力也较小。因此,社会主义运动会随着资本主义经济呈现周期性的波动。在经济繁荣时,社会主义运动基本只能局限于书本上的宣传,或者尝试用某种社会组织逐步代替资本主义经济结构(这种方法我们会在后面详细说明)。但是很难实现对于资本主义的根本打击。而在经济危机爆发的时候,大量的失业人群、对于就业前景迷茫的学生群体,是社会主义革命强有力的群众基础。群众对于经济体制的不满很容易爆发出来。
以二十一世纪20年代的欧洲为例: 
经济方面:深厚的工业历史积累,全球化和美元霸权使得经济危机的作用被放大。工业的历史积累使得革命成功后能够迅速的建立起工业体系,保证基础设施的的修复和工业体系的战后建立。
意识形态方面:经历了长时间的资本主义的发展,意识形态已经足够成熟,欧洲的民众已经适应了工业社会的生活方式,经历了物质丰富的商品社会,社会分工已经成熟。在这种长期摆脱了生物线贫困的生活环境中生长出的意识形态,已经逐渐褪去了对于物质贫乏的恐慌状态,更容易实现基层劳动者联合的社会形态。
2.社会主义武装运动的方法
欧洲的老牌资本主义国家拥有先进的武器装备。因此,全面的正面战场时几乎无法取胜。想要在武装斗争中取得胜利,只能发动市民(工人)直接夺取城市,直接夺取政权的方式实现武装斗争的胜利。或者通过长时间的罢工使得政府停摆,通过渗透进军队的内应,直接夺取军队的指挥权,进而实现武装斗争的胜利。总而言之,在这个时代,由于敌我武装力量的差距,社会主义武装力量要避免与资本主义武装力量的直接冲突。
2.1武装夺取政权的准备条件
要是想明确发动暴力革命的前提条件,首先要明确敌我武力方面的差距:政府拥有制式武器,装甲部队,绝对的空中优势。而革命党人所能做到的武装力量基本只有零星的枪械、少量的无人机。
由此我们可以看出来敌我力量的差距实在是太大了。所以长期的战争基本是不可能成功的。所以革命的策略非常清晰了:通过一次全国的突然的武装暴乱夺取政权,进而确定对于军队的领导地位,再通过军队建立国内秩序。
革命的第一步就是建立纪律严明的组织。在武装斗争中,纪律是决定革命中每个行动能否执行到位的根本。尤其是通过一次决定性的武装暴乱进行的革命,纪律决定了革命能否成功。
在建立组织后,我们要分析政治环境和国际形势。欧洲,尤其是西欧,都在北大西洋公约这个军事同盟中。所以,革命必须在几个国家中同时发动,才能避免与正规军的正面冲突。目前来看,英国、德国、法国、西班牙,是欧洲革命首要考虑的国家。其中,法、德作为传统资本主义国家,具有足够的工业经验积累。西班牙的地理位置可以保证对于地中海的控制。英国是大西洋上不沉的航母。这些是保证国家安全的重要基石。
这样看来,我们的目的就很明确了:联合英、法、德、西班牙的共产党,或者直接在这些国家建立组织,并将组织渗透进国家正规军,确保武装革命时有正规军支持。
2.2夺取政权后秩序的建立
这个新兴的社会主义政权的最大威胁,来自于美国。
3.公有制生产制度的框架
从社会主义经济的角度来说,传统的计划经济无法应对现代经济的复杂关系,而市场经济通过分散式的处理方式,虽不完美,但给出了对于复杂体系的解决方案——每个人的理性判断和对于利润的追求,让每个人追求自己的利益,从而实现整个体系的平衡与进步。本质上是将经济体制的不公平转移到几个公司上,让少数几个公司承担骂名,让大众承担风险和后果。但是,周期性的经济危机和整体经济的脱实向虚,不仅带来贫富分化、地缘政治问题,还揭示了一个重要的问题——在市场经济体系下,政府依旧需要对整个复杂的经济体系进行分析和再分配,并且由于政府的阶级性,这种再分配是注定不平衡的。也就是说,金融海啸会周期性的冲击经济体系,并且由于政府再分配的一次次的妥协,金融海啸最终会冲垮整个经济体系。
计划经济总是被描述成物资匮乏的,缺乏活力的。我们若是想构建新的经济体系,就不能只是批判市场经济的弊端,更要思考市场经济的运行特点——经济是双向的。过去的计划经济由中央指挥,但是缺少反馈,因此导致了生产不能满足需求的情况。所以,新的经济模式首先要考虑生产侧与消费侧的交流。
如果我们根据需求生产生活必需品,而不是利润,那么我们就可以缩小贫富两级在生活品质上的差异。
根据已有的社会主义实践经验,我们发现,社会主义经济的核心在于生产资料的公有制,而不是计划经济。计划经济只是生产资料公有制的一种实现方式。我们可以通过确保生产资料的公有制来实现社会主义经济,而不必拘泥于计划经济。
所以,我们可以通过开源硬件、开源软件、开源设计等方式,让大众掌握生产资料的使用权,从而实现生产资料的公有制。通过这种方式,我们可以让每个人都能参与到生产中来,而不是被动的接受资本主义的剥削。
基于此,我提出二十一世纪建立社会主义社会的基本架构:
基于中央集权政府和国有企业确保社会的基本运转和稳定;与此同时确保生产资料的公有制。人民自发联合形成的盈利组织(生产组织)可以存在,同时保证一切生产资料的公有制。这样保证了劳动者有自由发挥自身创造力。
中央集权的必要性体现在经济和政治两方面。经济方面,通过统领全国的国有企业调度资源,实现全国范围内的经济生活方面的平等。政治方面的中央集权则是保证社会主义制度的稳固,防止资本主义的卷土重来。
\end{document}
